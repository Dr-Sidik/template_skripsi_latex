%%%%%%%%%%%%%%%%%%%%%%%%%%%%%%%%%%%
%%%  Filename: thesis_template.tex
%%%  ---
%%%  Template for Master Thesis at DTETI UGM   		
%%%  Created using thesisdtetiugm.cls
%%%  --- 
%%%  Written by Canggih Puspo Wibowo
%%%  [canggihpw@gmail.com]
%%%%%%%%%%%%%%%%%%%%%%%%%%%%%%%%%%%

%% Use option "bahasa" or "english" 
%%    to change the basic language used
%% User option "bachelor", "master", or "doctoral"
%% 	  to change the degree
% \documentclass[<bachelor/master/doctoral>,<bahasa/english>]{thesisdtetiugm}
\documentclass[bachelor,bahasa]{thesisdtetiugm}
%======================================
% Information Input
%======================================
% Input author's name and ID number
\author{<<AUTHOR>>}{<<NIM>>}
% Input the thesis' title
\title{<<JUDUL SKRIPSI>>}
% Program and the head of the program
\program{<<NAMA PRODI>>}{<<Program coordinator>>}{<<NIP>>}
% Name of department head and NIP
\departmenthead{<<Head of the department>>}{<<NIP>>}
\major{<<Major>>}
\yearsubmit{<<TAHUN PENDADARAN>>}
\examdate{<<Exam date>>}
% Name of thesis supervisors/promotors
\addsupervisor{Dosen Pembimbing 1, S.T., M.Eng., PhD.}{<<NIP xxxxxx>>}
\addsupervisor{Dosen Pembimbing 2, S.T., M.Eng., PhD.}{<<NIP xxxxxx>>}
%\addsupervisor{<<Supervisor 3>>}{<<NIP>>}
% Name of examiners
%\addexaminer{<<Examiner 1>>}{<<NIP 1>>}
%\addexaminer{<<Examiner 2>>}{<<NIP 2>>}
%\addexaminer{<<Examiner 3>>}{<<NIP 3>>}
%\addexaminer{<<Examiner 4>>}{<<NIP 4>>}
%\addexaminer{<<Examiner 5>>}{<<NIP 5>>}
%\addexaminer{<<Examiner 6>>}{<<NIP 6>>}
%\addexaminer{<<Examiner 7>>}{<<NIP 7>>}
%\addexaminer{<<Examiner 8>>}{<<NIP 8>>}
%\addexaminer{<<Examiner 9>>}{<<NIP 9>>}

%======================================

% correct bad hyphenation here [example]
% \babelhyphenation[<<english/bahasa>>]{op-tical net-works semi-conduc-tor}
%% Uncomment block of code below to disable hyphenation
%\tolerance=1
%\emergencystretch=\maxdimen
%\hyphenpenalty=10000
%\hbadness=10000

\begin{document}
%======================================
% Create cover etc
%======================================

%---- COVER ----
%\printcover{sample/logougm.png}{Pendadaran/Tesis/Ringkasan Tesis*}
\printcover{sample/logougm.png}{Bachelor}
% *Choose one

%---- ENDORSEMENT PAGE ----
% Select endorsement page type. If you want to use your own PDF file,  
% 	use \printendorsementpdf, or if you want to use JPG file, use 
% 	\printendorsementjpg. Otherwise, use \printendorsement.
% 	Choose one only. Comment out unused command(s).
%
\cleardoublepage \phantomsection
\printendorsement
%\printendorsementpdf
%\printendorsementjpg{sample/scanned-endorsement.jpg}

%---- DEDICATION PAGE ----
\cleardoublepage \phantomsection
\chapterstatement{contents/statement/statement}
%\chapterstatementjpg{sample/scanned-statement.jpg}

\cleardoublepage \phantomsection
\chapterdedication{contents/dedication/dedication}

%---- STATEMENT PAGE ----
% Select statement page type. If you want to use your own JPG file,  
%	use \chapterstatementjpg{<your *.jpg file path>}. Otherwise, 
%	use \chapterstatement{contents/statement/statement}.
%	Choose one only. Comment out unused command(s).
%


%---- PREFACE PAGE ----
\cleardoublepage \phantomsection
\chapterpreface{contents/preface/preface}

%======================================
% Create Table of Contents, List of Figures, List of Tables
% <Do not change this part>
%======================================
\cleardoublepage \phantomsection
\thetoc
\onehalfspacing
\tableofcontents
\singlespacing
\cleardoublepage \phantomsection
\thelot
\listoftables
\cleardoublepage \phantomsection
\thelof
\listoffigures

%======================================

%---- NOMENCLATURE PAGE ----
\cleardoublepage \phantomsection
\chapternomenclature{contents/nomenclature/nomenclature}

%---- INTISARI PAGE----
\cleardoublepage \phantomsection
\chapterintisari{contents/abstract/intisari}

%---- ABSTRACT PAGE----
\cleardoublepage \phantomsection
\chapterabstract{contents/abstract/abstract}


%======================================



%======================================
%  MAIN TEXT
%======================================
\startmain
% You can change 
%    the filename and location of the files inputted
\cleardoublepage \phantomsection
\include{contents/chapter-1/chapter-1}
\cleardoublepage \phantomsection
\chapter{Tinjauan Pustaka dan Dasar Teori}

\section{Tinjauan Pustaka}

Berisi tugas akhir-tugas akhir terdahulu yang terkait dengan judul skripsi yang dilakukan. Hal ini meliputi skripsi, tesis, atau publikasi terdahulu yang terkait dengan judul skripsi yang diusulkan. Lakukan pembahasan secara sistemastis dengan menjelaskan masalah apa yang dilakukan oleh tugas akhir terdahulu, kontribusi yang dilakukan, serta analisis penulis terkait dengan keunggulan dan keterbatasan tugas akhir. 

Setelah membahas berbagai tugas akhir terdahulu, maka alangkah baiknya penulis melakukan rangkuman terutama terkait dengan peluang pengembangan atau tugas akhir yang akan dilakukan.


\section{Dasar Teori}

Berisi teori-teori yang menjadi dasar solusi atau produk hasil skripsi. Dasar teori pada umumnya diperoleh melalui buku referensi, publikasi tugas akhir, dan informasi web yang dapat dipertanggungjawabkan. Hindari penggunaan dasar teori melalui tautan wikipedia, surat kabar, atau portal berita.

\subsection{Pengenalan Aplikasi Permainan}

Proses pembuatan \textit{game} dimulai dari pembuatan \textit{game design document} dimana 
dokumen ini akan menjadi landasan pengembangan game tersebut serta menginformasikan gambaran keseluruhan game yang akan dibuat \cite{ferdiana2012agile}. \textcolor{red}{\textit{Catatan: apapun yang diambil dari tulisan orang lain harus disitasi seperti dicontohkan \cite{ferdiana2012agile}.}}

\begin{figure}[h]
	\centering
	\includegraphics[width=12cm]{contents/chapter-2/gambar-buatan-sendiri.png}
	\caption[Caption]{Contoh gambar \cite{lukito2016}}
        %\caption{Contoh gambar}
	\label{Fig:gambar-buatan-sendiri}
\end{figure}



\textit{Game design document} adalah sebuah bagian penting dalam pembuatan game baik itu elemen-elemen penyusunnya maupun proses pengembangannya. Game design yang telah dibuat, dijabarkan satu persatu mengenai tahapan dalam pembuatan game dan hasilnya disatukan dalam bentuk dokumentasi \textit{game design document} yang digunakan oleh \textit{developer} sebagai buku petunjuk bagaimana membuat \textit{game} \cite{lukito2016}.

Dalam buku \textit{Game Design Essentials} disebutkan \textit{game design document} merupakan metode yang menghubungkan elemen-elemen penyusun \textit{game}, baik itu \textit{art, sound, program, 
gameplay} sehingga semuanya terdokumentasi menjadi satu dan menjadi acuan bagi para \textit{developer} dalam membuat \textit{game} \cite{wibirama2013dual}. 

\subsection{Dasar Teori Lainnya}

\section{Analisis Perbandingan Metode}

Di dalam tinjauan pustaka hasil akhirnya adalah analisis secara kualitatif atau pun secara kuantitatif kelebihan dan kekurangan metode jika dikaitkan dengan masalah, batasan-batasan masalah dan solusi yang dinginkan. Analisis kuantitatif tidak wajib teapi mempunyai nilai tambah di dalam tugas akhir saudara. Bagian ini menjelaskan kenapa metode tersebut dipilih dan uraikan dengan lebih jelas metode pelaksanaan tugas akhir yang ingin Anda lakukan. 

\section{Pertanyaan Tugas Akhir (Jika Perlu)}

Pertanyaan tugas akhir bersifat opsional dan dapat ditambahkan untuk menekankan hal-hal yang hendak diketahui dari tugas akhir berdasar pada tujuan tugas akhir. Pertanyaan tugas akhir dikenal dengan RQ (\textit{Research Question}) dan harus memiliki keterkaitan dengan RO (\textit{Research Objective}). Satu RO dapat memiliki satu atau lebih dari satu RQ. 


\cleardoublepage \phantomsection
\include{contents/chapter-3/chapter-3}
\cleardoublepage \phantomsection
\include{contents/chapter-4/chapter-4}
\cleardoublepage \phantomsection
\include{contents/chapter-5/chapter-5}
\cleardoublepage \phantomsection
\include{contents/chapter-6/chapter-6}

%======================================

%======================================
%  References
%======================================
\cleardoublepage \phantomsection
\thereferences
% You can change 
%    the filename and location of the files inputted
\bibliography{references}

%Hapus bagian di bawah setelah tidak diperlukan
\begin{center}
	\textcolor{red}{
	Catatan: Daftar pustaka adalah apa yang dirujuk atau disitasi, bukan apa yang telah dibaca, jika tidak ada dalam sitasi maka tidak perlu dituliskan dalam daftar pustaka.}
\end{center}

%======================================

%======================================
%  Appendix
%======================================
% You can change 
%    the filename and location of the files inputted
%    use \chapterappendix for the first page of the appendix
%    use \chapterappendixadd for the next page

\appendix

\cleardoublepage \phantomsection
\chapterappendix{contents/appendix/appendix-isi-lampiran}
\chapterappendixadd{contents/appendix/appendix-latex}
\chapterappendixadd{contents/appendix/appendix-penulisan-referensi}
\chapterappendixadd{contents/appendix/appendix-code}




%======================================

\end{document}